\documentclass{beamer}
\usepackage{times}
\usepackage{hyperref}
\usepackage{enumitem}

\mode<presentation>
{
  \usetheme{UAB}
  \setbeamercovered{transparent}
  \setbeamertemplate{headline}{}
  \setbeamertemplate{blocks}[rounded]
  \setbeamertemplate{navigation symbols}{}
}

\DeclareMathOperator*{\argmin}{argmin}
\DeclareMathOperator*{\argmax}{argmax}

\author{Steven Bethard}
\subject{CS 499: Senior Capstone}


\title{Piracy}
\date{October~$15^{\text{th}}$,~2013}

\begin{document}

\begin{frame}
\titlepage
\end{frame}

\begin{frame}{Quiz: True or False?}

\begin{enumerate}[(A)]
\item<1> 27\% of US adults acquire most of their music by copying % no; 14% overall, 27% only for under 30
\item<1> Germans get a larger percent of their music from sharing % no; a smaller percent when controlling for collection size
\item<1-2> In Germany, creating copies of music for friends is legal
\item<1> Germans have smaller digital collections, but stream more music % no; they also stream less music
\item<1-2> People copy less when streaming services are available
\item<1> Most people are okay with sharing music/videos with friends % no; under 30 says yes, over 50 says no
\item<1-2> The majority of people support penalties for downloading
\item<1> The majority of people oppose blocking if legal content is blocked % no; US opposes but Germans (barely) approve
\item<1> In the US, 36\% of P2P users hide their IP addresses % no; 16% in US, 36% in Germany
\item<1-2> People are willing to pay $>$\$15 a month for legal file sharing
\end{enumerate}
\end{frame}

\begin{frame}{Discussion Questions}
\begin{itemize}
\item What most surprised you in this survey?
\bigskip
\item How can we get an accurate measure of underreporting?
% Tie to underreporting of weight:
% http://onlinelibrary.wiley.com/doi/10.1002/oby.20451/abstract
\bigskip
\item What should count as a friend for the purposes of sharing?
\end{itemize}
\end{frame}

\begin{frame}{Private Copying Laws}
\begin{description}
\item[Private copy:] ``any copy for non-commercial purposes made by a natural person for his/her own use''
\end{description}

Germany
% http://www.coreach-ipr.org/documents/CoReach%20Utrecht-Private%20copying%20in%20Germany.pdf
\begin{itemize}
\item 1965 -- Levies (collected from importer/manufacturer) from media
\item 1985 -- Levies from devices
\item 2003 -- Regardless of format (analog, digital)
\item 2003 -- No levies when media has DRM
\item 2008 -- Explicitly forbids ``evidently unlawful public sources''
\item 2009 -- ISPs must disclose copyright infringers
\end{itemize}

\bigskip
United States
\begin{itemize}
\item 1992 -- Audio Home Recording Act -- levies for audio devices/media
\item Subsequent laws: copyright exemption not extended to digital
\end{itemize}
\end{frame}

\begin{frame}{Stop Online Privacy Act of 2011}
\end{frame}

\begin{frame}{The German Pirate Party}
\end{frame}

\begin{frame}{Takedown of Megaupload}
\end{frame}

\begin{frame}{Anti-Counterfeiting Trade Agreement}
\end{frame}

\end{document}
