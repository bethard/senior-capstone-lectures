\documentclass{beamer}
\usepackage{times}
\usepackage{hyperref}
\usepackage{enumitem}

\mode<presentation>
{
  \usetheme{UAB}
  \setbeamercovered{transparent}
  \setbeamertemplate{headline}{}
  \setbeamertemplate{blocks}[rounded]
  \setbeamertemplate{navigation symbols}{}
}

\DeclareMathOperator*{\argmin}{argmin}
\DeclareMathOperator*{\argmax}{argmax}

\author{Steven Bethard}
\subject{CS 499: Senior Capstone}


\title{Patents}
\date{September~$19^{\text{th}}$,~2013}

\begin{document}

\begin{frame}
\titlepage
\end{frame}

\begin{frame}{Quiz}
Stallman believes the \textbf{best} argument against software patents is:
\begin{enumerate}[(A)]
\item<1-2> Software is essentially math, and math is not patentable
\item<1> Patents obfuscate an idea through the use of legal language
\item<1> Unknowingly violating patents is easy as they are hard to search
\item<1> Overturning patents is prohibitively expensive for small inventors
\end{enumerate}
\bigskip
Quinn believes the \textbf{worst} argument against software patents is:
\begin{enumerate}[(A)]
\item<1-2> Patents lock up the rights for a broad category of innovation
\item<1> Patents prevent improvements on existing technology
\item<1> Patents hold up innovation and economic growth
\item<1> The patent system is only affordable for large corporations
\end{enumerate}
\end{frame}

\begin{frame}{U.S. Patent Act}
Eligibility:
\begin{itemize}
\item ``any new and useful process, machine, manufacture, or composition of matter, or any new and useful improvement thereof''
\item useful $\approx$ has an intended purpose
\end{itemize}
Novelty and non-obviousness conditions:
\begin{itemize}
\item Not patented, published, sold or publicly known
\end{itemize}
America Invents Act (took effect March 16, 2013):
\begin{itemize}
\item First-inventor-to-file instead of first-to-invent
\end{itemize}
\bigskip
Result:
\begin{itemize}
\item Idea is your exclusive property for 20 years
\item Right to exclude others from making, using, selling, importing
\end{itemize}
\end{frame}

\end{document}
