\documentclass{beamer}
\usepackage{times}
\usepackage{hyperref}
\usepackage{enumitem}

\mode<presentation>
{
  \usetheme{UAB}
  \setbeamercovered{transparent}
  \setbeamertemplate{headline}{}
  \setbeamertemplate{blocks}[rounded]
  \setbeamertemplate{navigation symbols}{}
}

\DeclareMathOperator*{\argmin}{argmin}
\DeclareMathOperator*{\argmax}{argmax}

\author{Steven Bethard}
\subject{CS 499: Senior Capstone}


\title{Social Inequality}
\date{Nov~$19^{\text{th}}$,~2013}

\begin{document}

\begin{frame}
\titlepage
\end{frame}

\begin{frame}{Quiz}
The 2003 article by Jessica Brown reported that:
\begin{enumerate}[(A)]
\item<1> Of households making $<$\$15K, less than 5\% had internet access % No: 10%
\item<1> A low income white child was 2 times more likely to have internet access at home than a low income black child % No: 3 times
\item<1-2> Schools with 90\% minority children had 70\% higher student-to-computer ratios than average % 17-1 vs. 10-1
\item<1> Schools in high-income areas were twice as likely to have internet access as schools in low income areas % No: 75% vs. 55%
\end{enumerate}
\bigskip
The 2013 article by Jessica Goodman reported that:
\begin{enumerate}[(A)]
\item<1> 91\% of Americans have broadband at home % 71\%
\item<1> 66\% of 18-29 year-olds have a laptop; 57\% have a smartphone % reversed: 57% and 66%
\item<1-2> 98\% of the U.S. population has access to 3 Mbps downstream
%100 million people in the U.S. lived in areas where they had access to broadband, but did not subscribe
\item<1> $>$95\% of Fortune 500 companies require online job applications % 80%
\end{enumerate}
\end{frame}

% Have you met anyone who doesn't have internet access?
% Do you know anywhere you can go for free internet access?

% Consequences of Newsweek going all digital?
% http://www.theguardian.com/media-network/media-network-blog/2013/jan/02/end-newsweek-digital-evolution-rosenblum

% A High School Without Textbooks
% http://www.youtube.com/watch?v=UB2RXPqFgdI
\end{document}
