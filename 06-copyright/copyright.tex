% !TEX TS-program = pdflatexmk

\documentclass{beamer}
\usepackage{times}
\usepackage{hyperref}
\usepackage{enumitem}

\mode<presentation>
{
  \usetheme{UAB}
  \setbeamercovered{transparent}
  \setbeamertemplate{headline}{}
  \setbeamertemplate{blocks}[rounded]
  \setbeamertemplate{navigation symbols}{}
}

\DeclareMathOperator*{\argmin}{argmin}
\DeclareMathOperator*{\argmax}{argmax}

\author{Steven Bethard}
\subject{CS 499: Senior Capstone}

\setbeamercovered{invisible}

\title{Copyright}
\date{}

\begin{document}

\begin{frame}
\titlepage
\end{frame}

\begin{frame}{Quiz \hfill (Choose zero or more)}
Under the GNU manifesto, software companies could make money by:
\begin{enumerate}[(A)]
\item<1-2> distributing GNU software for a fee
\item<1-2> providing fee-based technical support
\item<1> charging when source code is copied
\item<1-2> making custom software for users
\end{enumerate}
\medskip
Stallman regrets writing
\begin{enumerate}[(A)]
\item<1-2> ``\ldots I am writing [GNU] so that I can give it away free to everyone\ldots''
\item<1-2> ``\ldots everyone will be able to obtain good system software free, just like air \ldots''
\item<1-2> ``Restricting copying is \ldots the most common basis [for business in software]\ldots''
\item<1-2> ``People who have studied the issue of intellectual property rights \ldots say that there is no intrinsic right\ldots''
\end{enumerate}
\end{frame}

\begin{frame}{Copyright}
Protection for original works of authorship:
\begin{itemize}
\item Literary, dramatic, musical, architectural, choreographic, etc.
\item \textbf{Not:} ideas, procedures, processes, principles, discoveries, slogans
\item Exclusive right to reproduce, distribute, perform/display
\item Lasts life of author + 70 years (in United States since 1978)
\end{itemize}
\bigskip
Berne Convention for the Protection of Literary and Artistic Works
\begin{itemize}
\item Created 9 September 1886 (United States joined in 1988)
\item International application of copyrights + minimum standards 
\item Copyright is automatic (may have reduced damages w/o notice)
\item Lasts at least 50 years
\item May allow ``fair use`` for criticism, comment, reporting, etc.
\end{itemize}
\end{frame}

\begin{frame}{GNU General Public License v3 \hfill\href{http://www.gnu.org/copyleft/gpl.html}{\beamergotobutton{GPL v3}}}
\begin{itemize}
\item[3.] Can I use GPL v3 libraries to implement Digital Rights Management (DRM) software? \\
\begin{itemize}
\item<2->Yes, but that software can then be legally cracked.
\end{itemize}
\item[4.] Can I sell someone else's GPL v3 code?
\begin{itemize}
\item<2->Yes.
\end{itemize}
\item[5.] Can I modify GPL v3 code and distribute under another license?
\begin{itemize}
\item<2-> No.
\end{itemize}
\item[5.] Can I distribute GPL v3 source and a source with an incompatible license in the same zip file?
\begin{itemize}
\item<2->Yes, if they are fully independent.
\end{itemize}
\item[6.] Can I modify GPL v3 sources and charge people for this code?
\begin{itemize}
\item<2-> No, though you may charge shipping costs.
\end{itemize}
\item[6.] Can I implement hardware restrictions that keep people from modifying GPL v3 code?
\begin{itemize}
\item<2-> Yes, but only if they prevent \emph{everyone} from modifying the code.
\end{itemize}
\item[13.] Can I use GPL v3 libraries to implement a web service and not distribute the sources?
\begin{itemize}
\item<2-> Yes. To prevent this, use the GNU Affero General Public License.
\end{itemize}
\end{itemize}
\end{frame}

\begin{frame}{BSD 2-clause license \hfill\href{http://opensource.org/licenses/BSD-2-Clause}{\beamergotobutton{BSD 2-clause}}}
Copyright (c) <YEAR>, <OWNER>\\
All rights reserved.\\
\medskip
Redistribution and use in source and binary forms, with or without modification, are permitted provided that the following conditions are met:\\
\begin{itemize}
\item Redistributions of source code must retain the above copyright notice, this list of conditions and the following disclaimer.
\item Redistributions in binary form must reproduce the above copyright notice, this list of conditions and the following disclaimer in the documentation and/or other materials provided with the distribution.
\end{itemize}
THIS SOFTWARE IS PROVIDED BY THE COPYRIGHT HOLDERS AND CONTRIBUTORS "AS IS" AND ANY EXPRESS OR IMPLIED WARRANTIES, INCLUDING, BUT NOT LIMITED TO, THE IMPLIED WARRANTIES OF MERCHANTABILITY AND FITNESS FOR A PARTICULAR PURPOSE ARE DISCLAIMED\ldots
\end{frame}

\begin{frame}{Apache License, Version 2.0 \hfill\href{http://www.apache.org/licenses/LICENSE-2.0.html}{\beamergotobutton{Apache v2}}}
\begin{itemize}
\item Grant of Copyright License (from all contributors)
\item Grant of Patent License (from all contributors)
\item Redistribution
\begin{itemize}
\item With or without modification (must identify modifications)
\item Must include license and other copyright/patent/etc. notices
\end{itemize}
\item Submission of Contributions (are all under license)
\item Trademarks (not granted by license)
\item Disclaimer of Warranty
\item Limitation of Liability
\item Accepting Warranty or Additional Liability
\end{itemize}
\end{frame}

\begin{frame}{Scenario: Patent infringement}
Microsoft claimed that at least 235 of its patents were violated by open source software, including parts of the Linux kernel.
In November 2006, licensing agreement was made between Microsoft and Novell (a major contributor to the Linux kernel).\\
\bigskip
How is this situation handled by:
\begin{itemize}
\item The GNU General Public License v3?
\item The BSD 2-clause license?
\item The Apache License, Version 2.0?
\end{itemize}
\end{frame}

\begin{frame}{Scenario: Combining code with multiple licenses}
Your company is writing software that uses several libraries that are under different licenses.
What license must you use for the overall project if the libraries include:
\begin{itemize}
\item GPL + BSD = \alt<2->{GPL}{?}
\item GPL + Apache2 = \alt<2->{GPL}{?}
\item BSD + Apache2 = \alt<2->{You decide}{?}
\item GPL + BSD + Apache2 = \alt<2->{GPL}{?}
\end{itemize}
\bigskip
\uncover<3->{You still must include all secondary licenses!}
\end{frame}

\end{document}
