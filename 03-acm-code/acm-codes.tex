\documentclass{beamer}
\usepackage{times}
\usepackage{hyperref}
\usepackage{enumitem}

\mode<presentation>
{
  \usetheme{UAB}
  \setbeamercovered{transparent}
  \setbeamertemplate{headline}{}
  \setbeamertemplate{blocks}[rounded]
  \setbeamertemplate{navigation symbols}{}
}

\DeclareMathOperator*{\argmin}{argmin}
\DeclareMathOperator*{\argmax}{argmax}

\author{Steven Bethard}
\subject{CS 499: Senior Capstone}


\title[ACM Code]{ACM Code of Ethics and Professional Conduct}
\date{September~$5^{\text{th}}$,~2013}

\begin{document}

\begin{frame}
\titlepage
\end{frame}

\begin{frame}{Quiz}
\begin{block}{1. The ACM Code of Ethics recommends that\ldots}
\begin{enumerate}[(A)]
\item<1> violations of the code should be reported to the ACM
\item<1-3> we provide critical reviews of the work of other professionals
\item<1> sources of code or ideas should be cited, if copyrighted/patented
\item<1> we challenge laws that interfere with workplace efficiency
\end{enumerate}
\end{block}
\begin{block}{2. The ACM Code of Ethics would support\ldots}
\begin{enumerate}[(A)]
\item<1-3> a consultant who recommends a company he openly holds stock in
\item<1-2> legal action against professionals who produced buggy software
\item<1-2> extra time after a project if users are unhappy with the interface
\item<1-2> software for an employment agency that sorts by race or gender
\end{enumerate}
\end{block}
\end{frame}

\begin{frame}{ACM Code of Ethics \href{http://www.acm.org/about/code-of-ethics}{\beamergotobutton{On the ACM webpage}}}
\begin{enumerate}
\item General Moral Imperatives
  \only<2>{
  \begin{enumerate}
  \item Contribute to society and human well-being.
  \item Avoid harm to others.
  \item Be honest and trustworthy.
  \item Be fair and take action not to discriminate.
  \item Honor property rights including copyrights and patent.
  \item Give proper credit for intellectual property.
  \item Respect the privacy of others.
  \item Honor confidentiality.
  \end{enumerate}
  }
\item More Specific Professional Responsibilities
  \only<3>{
  \begin{enumerate}
  \item Strive to achieve quality, effectiveness, dignity
  \item Acquire and maintain professional competence
  \item Know and respect existing laws
  \item Accept and provide professional review
  \item Give evaluations of systems and their impacts, risks
  \item Honor contracts, agreements, responsibilities
  \item Improve public understanding of computing
  \item Access resources only when authorized to do so
  \end{enumerate}
  }
\item Organizational Leadership Imperatives
  \only<4>{
  \begin{enumerate}
  \item Articulate social responsibilities and encourage acceptance
  \item Build systems that enhance the quality of life
  \item Define proper uses of an organization's resources
  \item Articulate user needs and validate system against them
  \item Support policies that protect the dignity of users
  \item Create opportunities to learn about computer systems
  \end{enumerate}
  }
\item Compliance with the Code
  \only<5>{
  \begin{enumerate}
  \item Uphold and promote the principles of this Code
  \item Treat violations as inconsistent with membership in the ACM
  \end{enumerate}
  }
\end{enumerate}
\end{frame}

\begin{frame}{IEEE Code of Ethics \href{http://www.ieee.org/about/corporate/governance/p7-8.html}{\beamergotobutton{On the IEEE webpage}}}
\begin{enumerate}
\only<1>{
\item to accept responsibility in making decisions consistent with the safety, health, and welfare of the public, and to disclose promptly factors that might endanger the public or the environment;
to avoid real or perceived conflicts of interest whenever possible, and to disclose them to affected parties when they do exist;
\item to be honest and realistic in stating claims or estimates based on available data;  
\item to reject bribery in all its forms;  
\item to improve the understanding of technology; its appropriate application, and potential consequences;  
}
\only<2>{
\setcounter{enumi}{4}
\item to maintain and improve our technical competence and to undertake technological tasks for others only if qualified by training or experience, or after full disclosure of pertinent limitations;  
\item to seek, accept, and offer honest criticism of technical work, to acknowledge and correct errors, and to credit properly the contributions of others;  
\item to treat fairly all persons regardless of such factors as race, religion, gender, disability, age, or national origin;  
\item to avoid injuring others, their property, reputation, or employment by false or malicious action;  
\item to assist colleagues and co-workers in their professional development and to support them in following this code of ethics.
}
\end{enumerate}
\end{frame}

\begin{frame}{NSPE Code of Ethics for Engineers \href{http://www.nspe.org/Ethics/CodeofEthics/}{\beamergotobutton{On the NSPE webpage}}}
\begin{enumerate}
\item Fundamental Canons
  \only<2>{
  \begin{enumerate}
  \item Hold paramount the safety, health, and welfare of the public.
  \item Perform services only in areas of their competence.
  \item Issue public statements only in an objective and truthful manner.
  \item Act for each employer or client as faithful agents or trustees.
  \item Avoid deceptive acts.
  \item Conduct themselves honorably, responsibly, ethically, and lawfully so as to enhance the honor, reputation, and usefulness of the profession.
  \end{enumerate}
  }
\item Rules of Practice
\item Professional Obligations
\end{enumerate}
\end{frame}

\begin{frame}{Comparing Ethical Codes: ACM vs. IEEE}
ACM: http://www.acm.org/about/code-of-ethics \\
IEEE: http://www.ieee.org/about/corporate/governance/p7-8.html
\bigskip
\begin{itemize}
\item Identify an element in the ACM code that is not explicitly in the IEEE code. Does some point in the IEEE code implicitly cover this?
\item Identify an element in the IEEE code that is not explicitly in the ACM code. Does some point in the ACM code implicitly cover this?
\end{itemize}
% IEEE vs ACM:
% Only explicit in IEEE: bribery
% Only explicit in ACM: intellectual property, privacy, confidentiality, laws
\end{frame}

\end{document}
