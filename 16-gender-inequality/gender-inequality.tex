\documentclass{beamer}
\usepackage{times}
\usepackage{hyperref}
\usepackage{enumitem}

\mode<presentation>
{
  \usetheme{UAB}
  \setbeamercovered{transparent}
  \setbeamertemplate{headline}{}
  \setbeamertemplate{blocks}[rounded]
  \setbeamertemplate{navigation symbols}{}
}

\DeclareMathOperator*{\argmin}{argmin}
\DeclareMathOperator*{\argmax}{argmax}

\author{Steven Bethard}
\subject{CS 499: Senior Capstone}


\title{Gender Inequality}
\date{Nov~$21^{\text{st}}$,~2013}

\begin{document}

\begin{frame}
\titlepage
\end{frame}

\begin{frame}{Quiz}
Campbell (2000) suggests that:
\begin{enumerate}[(A)]
\item<1> Males prefer challenge, debate and collaboration % No; challenge, debate and individual activities
\item<1> Commands like \texttt{grep}, \text{sed} and \text{awk} are gender biased % No; "killing a job", "fatal errors", "crashing"
\item<1> Men prefer ``rapport talk'' while women prefer ``report talk'' % No; the opposite
\item<1> Women were said to ``dominate'' a forum at 50\% of contributions % No; 30%
\item<1-2> Computer labs should schedule a ``women-only'' time
\end{enumerate}
\bigskip
In the PyCon event:
\begin{enumerate}[(A)]
\item<1> ``Mr. Hank'' made a sexual joke about ``forking'' a repository % No; only the ``dongle'' joke was sexual
\item<1> Ms. Richards's tweet violated the PyCon code of conduct % No; code of conduct was updated later
\item<1> SendGrid terminated the employment of ``Mr. Hank'' % No; Playhaven
\item<1-2> Anonymous DDoSed SendGrid in response
\item<1> Playhaven terminated the employment of Ms. Richards % No; SendGrid
\end{enumerate}
\end{frame}

\begin{frame}{Discussion of Campbell (2000)}
What points from Campbell (2000) are now out of date?\\[5em]
% games aimed at girls/women?
What points from Campbell (2000) are still relevant?
% teaching collaborative for women vs. teaching collaborative because it's required in jobs
\end{frame}

\begin{frame}{Learning Styles: Concepts and Evidence (2009)}
% http://dx.doi.org/10.1111/j.1539-6053.2009.01038.x
What is the theory of learning styles?
\pause
\begin{itemize}
\item Different people learn better from different teaching styles
\item E.g. visual vs. verbal
\end{itemize}
\pause
\bigskip
What is wrong with the following experiments?
\begin{itemize}
% students from each group not assigned different teaching styles
\item Survey students, asking what learning style they prefer.
\pause
% students not divided into groups by learning style
\item Randomly assign students to different teaching styles, measure test performance for each style.
\end{itemize}
\pause
\bigskip
Results of literature review:
\begin{itemize}
\item Different people prefer different styles
\item No evidence they learn better from that style
\end{itemize}
\end{frame}

\begin{frame}{Discussion of Cutler (2013)}
\begin{itemize}
\item What made the comments offensive?
% "Forking and Dongle Jokes Don?t Belong At Tech Conferences"
% http://news.rapgenius.com/Adria-richards-forking-and-dongle-jokes-dont-belong-at-tech-conferences-lyrics
% We were at a tech conference
% There was a job fair going on
% Women historically have felt unwelcome at tech conferences
% PyCon was making a special effort to be welcoming to women
% There were several women?s groups here (PyLadies, Women Who Code, CodeChix, Ada Initiative)
% He was wearing company logos and that meant his actions and words carried on their behalf
\bigskip
\item People complain on Twitter all the time. How was this different?
\bigskip
\item What factors did Playhaven and SendGrid have to consider before firing their employees?
\bigskip
\item How does this event affect perceptions of women in technology? Of men in technology?
\end{itemize}
\end{frame}

\begin{frame}{Women in Computing - Take 2 (2009)}
%http://cacm.acm.org/magazines/2009/2/19326-women-in-computing-take-2/fulltext
Stats on women in CS:
\begin{itemize}
\item[+] undergrad degrees: 7,000 in 1995 $\rightarrow$ 11,000 in 2005
\item[+] doctoral degrees: 16.5\% in 1997 $\rightarrow$ 19.8\% in 2005
\item[+] full professors: 5\% in 1995 $\rightarrow$ 10.9\% in 2007
\pause
\item[--] undergrad degrees: 37\% in 1985 $\rightarrow$ 22\% in 2005
\item[--] intention to major in CS: 2.8\% in 1985, 1.3\% in 1995, 0.4\% in 2006
\item[--] employed in math and CS: 33\% in 1984 $\rightarrow$ 27\% in 2004
\end{itemize}
\end{frame}

\begin{frame}{Women in Computing - Take 2 (2009)}
%http://cacm.acm.org/magazines/2009/2/19326-women-in-computing-take-2/fulltext
A sample of initiatives for women in technology:
\begin{itemize}
\item \href{http://girlsgotech.org}{Girls Scouts' Girls Go Tech}
% http://www.girlscoutsnorcal.org/pages/events/ggt.html
% http://vimeo.com/73735417
\item \href{http://engineering.union.edu/edge/}{Educating Girls for Engineering (EDGE) Workshop at Union College}
\item \href{http://women.acm.org/}{ACM Committee on Women in Computing}
\item \href{http://www.google.com/anitaborg/}{Google Anita Borg Memorial Scholarship}
\item \href{http://dx.doi.org/10.1145/1734263.1734281}{Harvey Mudd College course restructuring}
% Switched from Java to Python; split intro course into "standard" and "enrichment" tracks
% Offered trips for first year students to Grace Hopper Celebration of Women in Computing
% Undergraduate research experiences for sophomores
% Also: http://www.npr.org/blogs/alltechconsidered/2013/05/01/178810710/How-One-College-Is-Closing-The-Tech-Gender-Gap
\item Teaching pair programming (\href{http://cacm.acm.org/magazines/2006/8/5850/fulltext}{11.1\% $\rightarrow$ 46.3\% declare CS major})
\item \href{http://cra-w.org/dreu}{CRA-W's Distributed Research Experiences for Undergraduates}
\item National Science Foundation criteria 2: broadening participation of underrepresented groups in STEM
\end{itemize}
\end{frame}

\begin{frame}{GoldieBlox}
\href{http://www.youtube.com/watch?v=y-AtZfNU3zw}{GoldieBlox: Engineering toys for girls}
\begin{itemize}
\item What issue is the GoldieBlox founder trying to address?
\item How does she believe she's addressing that issue?
\item Why not just give girls Legos or other existing engineering toys?
\item How could we test if GoldieBlox has the desired effect?
\end{itemize}
\end{frame}

\end{document}
