% !TEX TS-program = pdflatexmk

\documentclass{beamer}
\usepackage{times}
\usepackage{hyperref}
\usepackage{enumitem}

\mode<presentation>
{
  \usetheme{UAB}
  \setbeamercovered{transparent}
  \setbeamertemplate{headline}{}
  \setbeamertemplate{blocks}[rounded]
  \setbeamertemplate{navigation symbols}{}
}

\DeclareMathOperator*{\argmin}{argmin}
\DeclareMathOperator*{\argmax}{argmax}

\author{Steven Bethard}
\subject{CS 499: Senior Capstone}


\title[Ethical Codes]{Ethical Codes}
\date{}

\begin{document}

\begin{frame}
\titlepage
\end{frame}

\begin{frame}{Quiz}

\begin{block}{1. A utilitarian ethical code is...}
\begin{enumerate}[(A)]
\item<1-3> $\displaystyle \argmax_{a \in actions} \left( \sum_{h \in humans} \left( happiness(h, a) - unhappiness(h, a)\right)\right)$
\item<1> $\displaystyle \argmax_{a \in actions} \left( \sum_{h \in humans} happiness(h, a)\right)$
\item<1> $\displaystyle \argmax_{a \in actions} \left( \sum_{h \in humans} \left( unhappiness(h, a) - happiness(h, a)\right)\right)$
\end{enumerate}
\end{block}
\begin{block}{2. Kant believed...}
\begin{enumerate}[(A)]
\item<1-2> We have a direct duty to avoid torturing animals
\item<1-2> Humans may be used as a means to an end
\item<1-3> Criminal punishment is a form of respect for the criminal
\end{enumerate}
\end{block}
\end{frame}

\begin{frame}{Utilitarianism}
Actions judged by impact on happiness of everyone
\begin{itemize}
\item Intent of action is irrelevant
\item Only consequences matter
\end{itemize}
\bigskip
Summary:
\begin{itemize}
\item $\displaystyle \argmax_{a \in actions} \left( \sum_{h \in humans} \left( happiness(h, a) - unhappiness(h, a)\right)\right)$
\end{itemize}
\end{frame}

\begin{frame}{Utilitarianism: History}
Jeremy Bentham, 1789
\begin{itemize}
\item \textit{Nature has placed mankind under the governance of two sovereign masters, pain and pleasure\ldots
By the principle of utility is meant that principle which approves or disapproves of every action whatsoever according to the tendency it appears to have to augment or diminish the happiness of the party whose interest is in question\ldots}
\item Pleasure/pain measured by: intensity, duration, certainty, propinquity, fecundity, purity, extent
\end{itemize}
\end{frame}

\begin{frame}{Utilitarianism: History}
John Stuart Mill, 1861
\begin{itemize}
\item On why happiness = good: \textit{\ldots we have not only all the proof which the case admits of, but all which it is possible to require, that happiness is a good: that each person's happiness is a good to that person, and the general happiness, therefore, a good to the aggregate of all persons.}
\end{itemize}
\pause
G. E. Moore, 1912
\begin{itemize}
\item On why happiness $\neq$ good: \textit{It involves our saying that, even if the total quantity of pleasure in each [world] was exactly equal, yet the fact that all the beings in the one possessed in addition knowledge of many different kinds and a full appreciation of all that was beautiful or worthy of love in their world, whereas none of the beings in the other possessed any of these things, would give us no reason whatever for preferring the former to the latter.}
\end{itemize}
\end{frame}

\begin{frame}{Example: Build a new Highway stretch}
Scenario:
  \begin{itemize}
  \item State may replace a curvy stretch of highway
  \item New highway segment 1 mile shorter (benefit)
  \item 150 houses would have to be removed (cost)
  \item Some wildlife habitat would be destroyed (cost)
  \end{itemize}
\pause
Analysis:
  \begin{itemize}
  \item[-] \$20 million to compensate homeowners
  \item[-] \$10 million to construct new highway
  \item[-] Lost wildlife habitat worth \$1 million
  \item[+] \$39 million savings in automobile driving costs
  \end{itemize}
\pause
Conclusion:
\begin{itemize}
\item \$39 million > \$31 million $\Rightarrow$ Building highway is a good action
\end{itemize}
\end{frame}

\begin{frame}{Evaluating Utilitarianism}
\begin{itemize}
  \item[+] Considers all stakeholders
  \item[+] Emphasizes the greatest good (i.e., happiness)
  \item[+] Cost/benefit analysis is a well-understood tool
  \item[+] Understandable ethical theory
\pause
  \item[-] Must be able to exactly quantify ``good''
  \item[-] Must be able to accurately predict all consequences
  \item[-] Majority outnumbers minority
  \item[-] Every situation has to be individually analyzed
\end{itemize}
\end{frame}

\begin{frame}{Categorical imperative}
Defined by Immanuel Kant (1724-1804)
\begin{itemize}
\item An (un)ethical action is always (un)ethical
\item Expected consequences of an action are morally neutral
\end{itemize}
\pause
\medskip
Formulation 1: \textit{Act only according to that maxim whereby you can, at the same time, will that it should become a universal law.}
\begin{itemize}
\item $\stackrel{?}{\approx}$ Golden Rule (\textit{Treat others how you wish to be treated})
\item Key difference: not just \emph{you}, but \emph{everyone}
\end{itemize}
\pause
\medskip
Formulation2: \textit{Act in such a way that you treat humanity, whether in your own person or in the person of any other, never
merely as a means to an end, but always at the same time as an end.}
\begin{itemize}
\item The end does not justify the means
\item Must allow others to make their own decisions
\end{itemize}
\end{frame}

\begin{frame}{Categorical imperative: Examples}
Stealing is unethical
\begin{itemize}
\item If everyone stole, property (and thus stealing) is meaningless
\item Stealing does not allow the owner to \emph{choose} to give item away
\end{itemize}
\medskip
\pause
Charity is ethical
\begin{itemize}
\item If no one ever helps others, no one can get help themselves
\item (Assumes that everyone needs help some time)
\end{itemize}
\medskip
\pause
Lying (or any deception) is unethical
\begin{itemize}
\item If everyone always lied, language is meaningless
\item Lying does not allow the lied-to to make their own decision
\end{itemize}
\pause
\medskip
\textbf{But should you tell a murderer where their victim is?}
\end{frame}

\begin{frame}{Categorical Imperative: Evaluation}
\begin{itemize}
\item[+] Universal maxims; easily applied to many situations
\item[+] Maxims are derived purely through reasoning
\item[+] Treats all persons as moral equals
\item[-] There are no exceptions to moral laws
\item[-] How do we resolve conflicts between rules?
\item[-] What are the underlying fundamental morals?
\end{itemize}
\end{frame}

\begin{frame}{Ethical Dilemma: Driverless Cars}
In four states (Nevada, Florida, California, Michigan) and the District of Columbia, it is now legal for Google to operate its driverless cars. Is it ethical to put driverless cars on the road? If not, what (if any) conditions would be necessary to make it ethical?
\bigskip
\begin{itemize}
\item Under utilitarianism?
\item Under the categorical imperative?
\end{itemize}
\end{frame}

\begin{frame}{Ethical Dilemma: Using wi-fi}
Your neighbor's wi-fi is not password protected. Is it ethical for you to use it? If not, what (if any) conditions would be necessary to make it ethical?
\bigskip
\begin{itemize}
\item Under utilitarianism?
\item Under the categorical imperative?
\end{itemize}
\end{frame}

\begin{frame}{Ethical Dilemma: Genetically Modified Crops}
A genetically modified strain of corn is more nutritious and more drought, stress and insect resistant than traditional corn. However, the patent for this strain is owned by a single company. Is it ethical to replace all our traditional corn crops with this genetically modified corn? If not, what (if any) conditions would be necessary to make it ethical?
\bigskip
\begin{itemize}
\item Under utilitarianism?
\item Under the categorical imperative?
\end{itemize}
\end{frame}


\begin{frame}{ACM Code of Ethics \href{http://www.acm.org/about/code-of-ethics}{\beamergotobutton{On the ACM webpage}}}
\begin{enumerate}
\item General Moral Imperatives
  \only<2>{
  \begin{enumerate}
  \item Contribute to society and human well-being.
  \item Avoid harm to others.
  \item Be honest and trustworthy.
  \item Be fair and take action not to discriminate.
  \item Honor property rights including copyrights and patent.
  \item Give proper credit for intellectual property.
  \item Respect the privacy of others.
  \item Honor confidentiality.
  \end{enumerate}
  }
\item More Specific Professional Responsibilities
  \only<3>{
  \begin{enumerate}
  \item Strive to achieve quality, effectiveness, dignity
  \item Acquire and maintain professional competence
  \item Know and respect existing laws
  \item Accept and provide professional review
  \item Give evaluations of systems and their impacts, risks
  \item Honor contracts, agreements, responsibilities
  \item Improve public understanding of computing
  \item Access resources only when authorized to do so
  \end{enumerate}
  }
\item Organizational Leadership Imperatives
  \only<4>{
  \begin{enumerate}
  \item Articulate social responsibilities and encourage acceptance
  \item Build systems that enhance the quality of life
  \item Define proper uses of an organization's resources
  \item Articulate user needs and validate system against them
  \item Support policies that protect the dignity of users
  \item Create opportunities to learn about computer systems
  \end{enumerate}
  }
\item Compliance with the Code
  \only<5>{
  \begin{enumerate}
  \item Uphold and promote the principles of this Code
  \item Treat violations as inconsistent with membership in the ACM
  \end{enumerate}
  }
\end{enumerate}
\end{frame}

\begin{frame}{IEEE Code of Ethics \href{http://www.ieee.org/about/corporate/governance/p7-8.html}{\beamergotobutton{On the IEEE webpage}}}
\begin{enumerate}
\only<1>{
\item to accept responsibility in making decisions consistent with the safety, health, and welfare of the public, and to disclose promptly factors that might endanger the public or the environment;
to avoid real or perceived conflicts of interest whenever possible, and to disclose them to affected parties when they do exist;
\item to be honest and realistic in stating claims or estimates based on available data;  
\item to reject bribery in all its forms;  
\item to improve the understanding of technology; its appropriate application, and potential consequences;  
}
\only<2>{
\setcounter{enumi}{4}
\item to maintain and improve our technical competence and to undertake technological tasks for others only if qualified by training or experience, or after full disclosure of pertinent limitations;  
\item to seek, accept, and offer honest criticism of technical work, to acknowledge and correct errors, and to credit properly the contributions of others;  
\item to treat fairly all persons regardless of such factors as race, religion, gender, disability, age, or national origin;  
\item to avoid injuring others, their property, reputation, or employment by false or malicious action;  
\item to assist colleagues and co-workers in their professional development and to support them in following this code of ethics.
}
\end{enumerate}
\end{frame}

\begin{frame}{NSPE Code of Ethics for Engineers \href{http://www.nspe.org/Ethics/CodeofEthics/index.html}{\beamergotobutton{On the NSPE webpage}}}
\begin{enumerate}
\item Fundamental Canons
  \only<2>{
  \begin{enumerate}
  \item Hold paramount the safety, health, and welfare of the public.
  \item Perform services only in areas of their competence.
  \item Issue public statements only in an objective and truthful manner.
  \item Act for each employer or client as faithful agents or trustees.
  \item Avoid deceptive acts.
  \item Conduct themselves honorably, responsibly, ethically, and lawfully so as to enhance the honor, reputation, and usefulness of the profession.
  \end{enumerate}
  }
\item Rules of Practice
\item Professional Obligations
\end{enumerate}
\end{frame}

\begin{frame}{Comparing Ethical Codes: ACM vs. IEEE}
ACM: http://www.acm.org/about/code-of-ethics \\
IEEE: http://www.ieee.org/about/corporate/governance/p7-8.html
\bigskip
\begin{itemize}
\item Identify an element in the ACM code that is not explicitly in the IEEE code. Does some point in the IEEE code implicitly cover this?
\item Identify an element in the IEEE code that is not explicitly in the ACM code. Does some point in the ACM code implicitly cover this?
\end{itemize}
% IEEE vs ACM:
% Only explicit in IEEE: bribery
% Only explicit in ACM: intellectual property, privacy, confidentiality, laws
\end{frame}

\begin{frame}{Case Study: Copying Files}
You are a computer system manager. An employee is out sick and another employee requests that you copy all files from the sick person's computer to his so he can do some work. \\
\bigskip
\textbf{ACM Code of Ethics?}
\begin{itemize}
\item<2-> 1.7 Respect the privacy of others.
\item<2-> 2.8 Access computing and communication resources only when authorized to do so.
\item<2-> 3.3 Acknowledge and support proper and authorized uses of an organization's computing and communication resources.
\end{itemize}
\end{frame}

\begin{frame}{Case Study: Computer-Controlled Laser for Tumors}
Your team is working on a computer-controlled laser device for treating cancerous tumors. The computer controls direction, intensity, and timing of the beam that destroys the tumor. Various delays have put the project behind schedule, and the deadline is approaching. There will not be time to complete all the planned testing. The system has been functioning properly in the routine treatment scenarios that have been tested so far. You are the project manager, and you are considering whether to deliver the system on time, while continuing testing, and to make patches if bugs are found. \\
\bigskip
\textbf{ACM Code of Ethics?}
\end{frame}

\begin{frame}{Case Study: Disabling Copy-Protection}
A small company offers you a programming job. You are to work on a new version of its software product that can disable copy-protection and other access controls on electronic books, allowing them to be displayed by the software. The program enables buyers of e-books to read their e-books on a variety of hardware devices (fair uses). However, customers could also use the program to make unauthorized copies of copyrighted books. \\
\bigskip
\textbf{ACM Code of Ethics?}
\end{frame}

\begin{frame}{Case Study: Benign Virus}
You are trying to study how computer viruses spread. You create a computer virus that is capable of infecting Windows PCs via email attachments. When the virus infects a PC it performs the following actions.
\begin{enumerate}
\item Alerts the user of its presence on the infected PC
\item Explains its intended purpose and the details of the experiment
\item Allows the user of the infected PC to delete the virus immediately
\item If the user allows the virus to remain:
  \begin{enumerate}
  \item Causes no damage to the infected PC
  \item Attempts to infect other PCs via email attachments
  \item Disables and deletes itself after 1 week
  \end{enumerate}
\end{enumerate}
\bigskip
\textbf{ACM Code of Ethics?}
\end{frame}

\end{document}
