% !TEX TS-program = pdflatexmk

\documentclass{beamer}
\usepackage{times}
\usepackage{hyperref}
\usepackage{enumitem}

\mode<presentation>
{
  \usetheme{UAB}
  \setbeamercovered{transparent}
  \setbeamertemplate{headline}{}
  \setbeamertemplate{blocks}[rounded]
  \setbeamertemplate{navigation symbols}{}
}

\DeclareMathOperator*{\argmin}{argmin}
\DeclareMathOperator*{\argmax}{argmax}

\author{Steven Bethard}
\subject{CS 499: Senior Capstone}


\title{Piracy}
\date{}

\begin{document}

\begin{frame}
\titlepage
\end{frame}

\begin{frame}{Quiz: True or False?}

\begin{enumerate}[(A)]
\item<1> 27\% of US adults acquire most of their music by copying % no; 14% overall, 27% only for under 30
\item<1-2> In Germany, creating copies of music for friends is legal
\item<1> Germans have smaller digital collections, but stream more music % no; they also stream less music
\item<1-2> People copy less when streaming services are available
\item<1> Most people are okay with sharing music/videos with friends % no; under 30 says yes, over 50 says no
\item<1-2> The majority of people support penalties for downloading
\item<1> The majority of people oppose blocking if legal content is blocked % no; US opposes but Germans (barely) approve
\item<1> In the US, 36\% of P2P users hide their IP addresses % no; 16% in US, 36% in Germany
\item<1-2> People are willing to pay $>$\$15 a month for legal file sharing
\end{enumerate}
\end{frame}

\begin{frame}{Discussion Questions}
\begin{itemize}
\item How big of a problem is underreporting? Is there a way to get an accurate measurement?
% Tie to underreporting of weight:
% http://onlinelibrary.wiley.com/doi/10.1002/oby.20451/abstract
\pause
\bigskip
\item How do you weigh the tradeoffs between privacy and copyright enforcement?
\pause
\bigskip
\item If it's ethical to share songs with your family, is it ethical to share them with your friends? With your Facebook friends?
\pause
\bigskip
\item Among digital file owners, P2P users download more files, but also buy more digital music. Does this influence the ethical concerns?
\pause
\bigskip
\item 8-9\% of people watch TV/movies via unauthorized streaming sites. What are the ethical issues for the stream host? For the viewer?
\end{itemize}
\end{frame}

\begin{frame}{Private Copying Laws}
\begin{description}
\item[Private copy:] ``any copy for non-commercial purposes made by a natural person for his/her own use''
\end{description}

Germany
% http://www.coreach-ipr.org/documents/CoReach%20Utrecht-Private%20copying%20in%20Germany.pdf
\begin{itemize}
\item 1965 -- Levies (collected from importer/manufacturer) from media
\item 1985 -- Levies from devices
\item 2003 -- Regardless of format (analog, digital)
\item 2003 -- No levies when media has DRM
\item 2008 -- Explicitly forbids ``evidently unlawful public sources''
\item 2009 -- ISPs must disclose copyright infringers
\end{itemize}

\bigskip
United States
\begin{itemize}
\item 1992 -- Audio Home Recording Act -- levies for audio devices/media
\item Subsequent laws: copyright exemption not extended to digital
\end{itemize}
\end{frame}

\begin{frame}{Swedish Pirate Party (Piratpartiet)}
Principles:
\begin{itemize}
\item Reform of copyright: 5 years, non-commercial exemption, no DRM
\item Reform of patent: no patents on life or software
\item Civil rights: privacy, free speech, transparent government
\end{itemize}
History:
\begin{itemize}
\item 2006 Jan -- Founded
\item 2006 May -- Police raid of The Pirate Bay increases membership
\item 2006 Sep -- 0.63\% votes, third largest outside parliament
\item 2009 Apr -- The Pirate Bay trial verdict: guilty, 1 year + fine
\item 2009 May -- Third largest party in Sweden
\item 2009 Jun -- 7.13\% Swedish votes for European parliament: 2 seats
\item 2011 Sep -- German version gets 9\% vote, state parliament seat
\end{itemize}
\end{frame}

\begin{frame}{Stop Online Privacy Act of 2011}
%http://sopastrike.com/timeline
\pause
\begin{itemize}
\item 2011 Oct 26 -- Proposed
\begin{itemize}
\item Court orders against non-US infringers to:
\begin{itemize}
\item Bar search engines from linking
\item Require ISPs/ad networks to block access
\end{itemize}
\item Copyright holder contact ISPs, etc. directly
\item Also increases penalties for streaming video
\item If ISPs/ad networks don't comply, they may be liable
\end{itemize}
\item 2011 Nov 16 -- 6000+ websites post self-censor logos in opposition
\item 2012 Jan 18 -- 115,000+ websites: Google, Wikipedia, Reddit\ldots
\item 2012 Jan 20 -- Withdrawn
\end{itemize}
\end{frame}

\begin{frame}{Anti-Counterfeiting Trade Agreement}
%http://www.techdirt.com/articles/20120124/11270917527/
\pause
\begin{itemize}
\item 2006 -- Developed by Japan and US, others join in 2006 -- 2008
\item 2010 -- Final draft released
\begin{itemize}
\item ACTA committee, outside of WTO and WIPO
\item Mostly consistent with current US law
\item ``Aiding and abetting'' $\Rightarrow$ criminal (not civil) charges
\item ``Indirect economic advantage'' $\Rightarrow$ commercial scale
\item Governments could not break patents for health reasons
\item Penalties should ``include imprisonment as well as monetary fines''
\end{itemize}
\item 2011 Oct -- Signing ceremony (8 countries, later 31)
\item 2012 Jan -- European Parliament's rapporteur resigns over ACTA
\item 2012 Feb -- Protests in 200+ European cities
\item 2012 Oct -- Japan ratifies treaty (need 6 ratifications)
\item 2012 Dec -- European union rejects treaty (92\%)
\end{itemize}
\end{frame}

\begin{frame}{Megaupload}
Megaupload file hosting
\begin{itemize}
\item Over 50,000,000 visitors per day
\item Claims to ``swiftly process legitimate DMCA takedown notices''
\item But multiple uploads link to same file, takedown applied per link
\end{itemize}
History:
\begin{itemize}
\item 2005 -- Company established in Hong Kong
\item 2009 -- Users with Hong Kong IP addresses banned from site
\item 2010 May -- Blocked in Saudi Arabia and UAE
\item 2011 Jun -- Blocked or throttled by Malaysian ISPs
\item 2011 Dec -- Endorsement song; Universal Music Group vs. YouTube
\item 2012 Jan -- US prosecutors seize and shut down, New Zealand arrests executives at US request, Hong Kong freezes assets
\item 2013 Jan -- Relaunch as ``Mega''
\end{itemize}
\end{frame}

\begin{frame}{Copyright Alert System (``Six Strikes'')}
Implementation began February 2013
\begin{itemize}
\item Recording industry: MPAA, IFTA, RIAA, A2IM
\item ISPs: AT\&T, Cablevision, Time Warner, Verizon, Comcast
\item MarkMonitor will monitor BitTorrent uploads
\item Suspected infringement $\Rightarrow$ ISP notified
\item Alerts to infringers
\begin{enumerate}
\item[1-2] Notice and educational materials
\item[3-4] Notice that requires user to acknowledge receipt
\item[5] ISP may take ``mitigation'' measures
\item[6] ISP must take ``mitigation'' measures
\item[???] ISPs must terminate repeat copyright infringer accounts
\end{enumerate}
\item Appeal possible (\$35) when mitigation measure to be imposed
\begin{itemize}
\item Burden of proof on subscriber
\item Case judged by American Arbitration Association
\item Unsecured wireless router is valid appeal, but only once
\end{itemize}
\end{itemize}
\end{frame}

\end{document}
