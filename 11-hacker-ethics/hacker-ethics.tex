\documentclass{beamer}
\usepackage{times}
\usepackage{hyperref}
\usepackage{enumitem}

\mode<presentation>
{
  \usetheme{UAB}
  \setbeamercovered{transparent}
  \setbeamertemplate{headline}{}
  \setbeamertemplate{blocks}[rounded]
  \setbeamertemplate{navigation symbols}{}
}

\DeclareMathOperator*{\argmin}{argmin}
\DeclareMathOperator*{\argmax}{argmax}

\author{Steven Bethard}
\subject{CS 499: Senior Capstone}

\usepackage{listings}
\lstset{%
basicstyle=\footnotesize\ttfamily\color{black},
commentstyle = \ttfamily\color{red},
keywordstyle=\ttfamily\color{blue},
stringstyle=\color{green!50!black},
showstringspaces=false}


\title{Hacker Ethics}
\date{October~$17^{\text{th}}$,~2013}

\begin{document}

\begin{frame}
\titlepage
\end{frame}

\begin{frame}{Quiz}
Spafford believes the biggest problem with utilitarianism is:
\begin{enumerate}[(A)]
\item<1> it would justify beheading 100 smokers on live television
\item<1> ``it isn't whether you win or lose, it's how you play the game''
\item<1-2> taking actions requires knowing the full scope of the results
\item<1> it results in disregard for due process and civil rights
\end{enumerate}
\bigskip
Spafford suggests that:
\begin{enumerate}[(A)]
\item<1-2> information could be truly free in a world of only ethical people
\item<1> break-ins can serve as a form of security flaw reporting % discuss: costs for users, vendors
\item<1> critical feedback is gained by hacking, but not CS theory education % no; neither
\item<1> hacking to prevent data misuse occasionally deters ``Big Brother''
\end{enumerate}
\end{frame}

\begin{frame}{Attacking the D.C. Internet Voting System}
Wolchok, Wustrow, Isabel, Halderman (2012)
\begin{itemize}
\item D.C. Digital Vote-By-Mail System:
\begin{itemize}
\item Ruby on Rails + Apache server + MySQL database
\item Voters fill out and upload PDFs
\end{itemize}
\item Attack methodology
\begin{itemize}
\item Read source code
\item Focus on voter login, ballot upload, database use, network activity
\end{itemize}
\item Mock election held before real election
\item Result: internet voting not used in real election
\end{itemize}
\end{frame}

\begin{frame}[fragile]{D.C. Internet Voting System Vulnerabilities}
Shell Injection (from using old copy from PaperClip library):
\begin{itemize}
\item[]
\begin{lstlisting}[language=Ruby]
run("gpg" , "--trust-model always
    -o \"#{File.expand_path(dst.path)}\" -e
    -r \"#{@recipient}\" \"#{File .expand_path(src .path)}\"")
\end{lstlisting}
\pause
\item[] Name your file ``foo.\$(cmd)''
\end{itemize}
\bigskip
\pause
Attacks:
\begin{itemize}
\item Gained key used for encrypting ballots
\item Gained access to database by checking bash history
\item Stolen key allowed replacing all existing ballots on system
\item Found pre-encryption version of past ballots in /tmp
\item Modified function to publicly post any future uploaded ballots
\item Modified server logs to hide activity
\end{itemize}
\end{frame}

\begin{frame}{More D.C. Internet Voting System Vulnerabilities}
\begin{itemize}
\item Credentials, session key, identical to those in public repository
\item Session IDs not randomized; sequential from 1
\item Application user had permissions to modify code
\item Version of Linux had known local root exploit
\item Public network block: router, gateway, webcams, terminal server
\item Default login and password in terminal server
\end{itemize}
\end{frame}

\begin{frame}{Was Hacking the D.C. Internet Voting System Ethical?}
\begin{itemize}
\item To Spafford? To a utilitarian?
\bigskip
\item Partially ethical? Which parts?
\end{itemize}
\end{frame}

\begin{frame}{Digital Carjackers}
\href{https://www.youtube.com/watch?v=oqe6S6m73Zw}{\beamergotobutton{Digital Carjackers Show Off New Attacks}} \\
\bigskip
Charlie Miller (security engineer, Twitter) \\
Chris Valasek (director of security intelligence, IOActive)
\begin{itemize}
\item Took control of Car Access Network (CAN)
\item Required physical access to car (not wireless)
%\item Code released http://blog.ioactive.com/2013/08/car-hacking-content.html
\end{itemize}
\pause
\bigskip
Is this kind of ``White Hat'' hacking ethical?
\begin{itemize}
\item To Spafford? To a utilitarian?
\item Partially ethical? Which parts?
\end{itemize}
\end{frame}

\end{document}
