\documentclass{beamer}
\usepackage{times}
\usepackage{hyperref}
\usepackage{enumitem}

\mode<presentation>
{
  \usetheme{UAB}
  \setbeamercovered{transparent}
  \setbeamertemplate{headline}{}
  \setbeamertemplate{blocks}[rounded]
  \setbeamertemplate{navigation symbols}{}
}

\DeclareMathOperator*{\argmin}{argmin}
\DeclareMathOperator*{\argmax}{argmax}

\author{Steven Bethard}
\subject{CS 499: Senior Capstone}

\usepackage{listings}
\lstset{%
basicstyle=\footnotesize\ttfamily\color{black},
commentstyle = \ttfamily\color{red},
keywordstyle=\ttfamily\color{blue},
stringstyle=\color{green!50!black},
showstringspaces=false}


\title{90s Hacker Ethics}
\date{October~$29^{\text{th}}$,~2013}

\begin{document}

\begin{frame}
\titlepage
\end{frame}

\begin{frame}{Quiz}
\begin{columns}[T]
\begin{column}{0.485\textwidth}
The key principle of the 60s Hacker Ethic is:
\begin{enumerate}[(A)]
\item<1> Hands on imperative
\item<1-2> Information wants to be free % freedom of movement vs. change/evolution vs. cost
\item<1> Mistrust authority
\item<1> No bogus criteria
\item<1> You can create truth and beauty on a computer
\item<1> Computers can change your life for the better
\end{enumerate}
\end{column}
\begin{column}{0.485\textwidth}
The key principle of the 90s Hacker Ethic is:
\begin{enumerate}[(A)]
\item<1> Above all else, do no harm
\item<1> Protect privacy
\item<1> Waste not, want not
\item<1> Exceed limitations
\item<1-2> The communicational imperative % questionable, but if you have to pick one...
\item<1> Leave no traces
\item<1> Share!
\item<1> Self defense
\item<1> Hacking helps security
\item<1> Trust, but test!
\end{enumerate}
\end{column}
\end{columns}
% Protect privacy vs. Information wants to be free
% Share! == Information wants to be free
\end{frame}

\begin{frame}{Terminology}
\begin{description}
\item[Cracking] \uncover<2->{Breaking copy protection on software}
\item[Phreaking] \uncover<2->{Obtaining toll free calls, e.g. via 2600 Hz tone}
% discovered by Joe Engressia, blind 7-year-old
% last multi-frequency exchange in the contiguous US disabled in 2006
\item[Virus] \uncover<2->{Malware attached to executable, spreads when run}
\item[Worm] \uncover<2->{Malware exploiting weak security, spreads independently}
\item[Trojan] \uncover<2->{Malware masquerading as legitimate, may open backdoor}
\item[Logic bomb] \uncover<2->{Code that becomes malicious when conditions are met}
\item[Warez] \uncover<2->{Pirated software, books, movies, etc.}
\item[Narc] \uncover<2->{Informant, turns in other hackers; from ``Narcotics agent''}
\end{description}
\end{frame}

\begin{frame}{Changes since Ethic for 90s Hackers was written?}
``Nowadays, few software vendors use copy protection''\\[1em]
\pause
``distribution of algorithms that make use of [strong encryption] is a felony''\\[1em]
% as late as 1992, cryptography was on the U.S. Munitions List as an Auxiliary Military Technology
\pause
``the NSA and FBI cannot break strong encryption''\\[1em]
\pause
``businesses keep the true state of the art equipment away from the people''\\[1em]
\pause
``Tools like the PC are said to move power away from large organizations (who use mainframes) and put them in the hands of the 'little guy' user.''\\[1em]
\pause
``why invest in four or five similar programs if we aren't sure which best suits our needs?''
\end{frame}

\begin{frame}{Reevaluating Ethical Scenarios via the Hacker Ethic}
Which elements of the hacker ethic would or would not support:\\[1em]
\begin{itemize}
\item ``Attacking the Washington D.C. Internet Voting System''
\begin{itemize}
\item<2-> Ran commands by uploading carefully named PDFs
\item<2-> Publicly posted uploaded ballots
\item<2-> Modified server logs to hide activity
\item<2-> Accessed webcams in server room
\end{itemize}
\item ``Digital Carjackers Show Off New Attacks''
\begin{itemize}
\item<2-> Took control of Car Access Network (CAN)
\item<2-> Caused vehicle to break, turn, shut off, etc.
\end{itemize}
\end{itemize}
\end{frame}

\begin{frame}{Certified Ethical Hacker}
Introduced by Council of Electronic Commerce Consultants, 2003
% http://www.eccouncil.org/Support/awards-and-recognition
\begin{itemize}
\item knows how to look for weaknesses and vulnerabilities
\item uses same knowledge and tools as malicious hacker
\item in a lawful and legitimate manner to assess security
\end{itemize}
\pause
\bigskip
How would ``90s Hackers'' react to this?
\end{frame}

\begin{frame}{NSA vs. Encryption}
% http://www.propublica.org/article/the-nsas-secret-campaign-to-crack-undermine-internet-encryption
% http://www.theguardian.com/world/2013/sep/05/nsa-how-to-remain-secure-surveillance
Report on NSA activities in New York Times and Pro Publica, Sep 2013:
\begin{itemize}
\item ``For the past decade, N.S.A. has led an aggressive, multipronged effort to break widely used Internet encryption technologies'' (2010)
\item custom-built, superfast computers to break codes
\item exploiting existing security flaws
\item hacking into computers before encryption or after decryption
\item getting technology companies to build backdoors
\item maintaining a database of commercial encryption keys
\item planted vulnerabilities in a 2006 security standard
\end{itemize}
\bigskip
NSA response: ``the fact that NSA's mission includes deciphering enciphered communications is not a secret, and is not news''
\end{frame}

\end{document}
