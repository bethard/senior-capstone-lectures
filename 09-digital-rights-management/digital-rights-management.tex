\documentclass{beamer}
\usepackage{times}
\usepackage{hyperref}
\usepackage{enumitem}

\mode<presentation>
{
  \usetheme{UAB}
  \setbeamercovered{transparent}
  \setbeamertemplate{headline}{}
  \setbeamertemplate{blocks}[rounded]
  \setbeamertemplate{navigation symbols}{}
}

\DeclareMathOperator*{\argmin}{argmin}
\DeclareMathOperator*{\argmax}{argmax}

\author{Steven Bethard}
\subject{CS 499: Senior Capstone}

\usepackage{listings}
\lstdefinelanguage{W3C}{
  keywords={attribute, readonly, void, Constructor, interface, partial, static, bool},
  keywordstyle=\color{blue},
  ndkeywords={HTMLSourceElement, HTMLMediaElement, EventHandler, MediaKeys, DOMString, Uint8Array, MediaKeySession, MediaKeyError, EventTarget},
  ndkeywordstyle=\color{green!50!black},
  morecomment=[l]{//},
}
\lstset{language=W3C}

\title{Open Source Software Development}
\date{October~$3^{\text{rd}}$,~2013}

\begin{document}

\begin{frame}
\titlepage
\end{frame}

\begin{frame}{Quiz}
Cory Doctorow believes that:
\begin{enumerate}[(A)]
\item<1> DRMs aim to enforce copyright by restricting downloads of assets % no, by preventing copying (primarily)
\item<1> In the DRM threat model, attacker and sender are the same person % recipient, not sender
\item<1> The main failure of DRM systems is that hackers can ``crack'' them % no, that DRM-free versions are available
\item<1-2> DRM to erase tax records assumes you both do and don't trust IRS
\end{enumerate}
\bigskip
Cory Doctorow explains that:
\begin{enumerate}[(A)]
\item<1-2> Authorized Domains let DRM systems define valid user groups
\item<1> Song DRM restrictions are fixed based on the time of download
\item<1> DRM systems allow exceptions for the creation of parodies
\item<1> DRM regional controls aim to enforce copyright restrictions
\end{enumerate}
\end{frame}

\begin{frame}{A brief history of Musical Digital Rights Managment}
% Inspired by http://opensource.com/life/11/11/drm-graveyard-brief-history-digital-rights-management-music
\begin{itemize}
\item Dec 1996: World Intellectual Property Organization Treaty
%Member states must enact laws against DRM circumvention
\item May 1998: Digital Millennium Copyright Act
%Criminalizes production/dissemination of DRM cracking software
%Reverse engineering for interoperability okay
\item Oct 2001: ``Beale Screamer'' cracks Microsoft Windows Media DRM
\item Apr 2003: iTunes store launches with FairPlay DRM
\item Nov 2003: Jon Lech Johansen cracks FairPlay
\item Oct 2005: Discovery of Sony CD rootkit in SecuROM DRM
\item Jul 2006: eMusic, DRM-free, is 2$^{\text{nd}}$ largest digital music service
\item Feb 2007: Steve Jobs, ``\href{http://www.apple.com/ca/hotnews/thoughtsonmusic/}{Thoughts on Music}''
\item Mar 2007: \href{http://www.michaelgeist.ca/content/view/1826/125/}{DMCA author Bruce Lehman says DRM unsuccessful}
\item Apr 2007: EMI music becomes DRM-free on iTunes
\item May 2007: Amazon launches DRM-free music sales
\item Sep 2008: WalMart shuts down DRM system (after only 5 months)
\item Jan 2009: Apple gets DRM-free from all 4 major music companies
\end{itemize}
\end{frame}

\begin{frame}{A brief history of Video Digital Rights Managment}
\begin{itemize}
\item 1996: Content Scrambling System (CSS): encrypted DVDs
\item 1999: John Lech Johansen cracks CSS (DeCSS) for Linux
\item 1999: Windows Media DRM: rights management language
\item 2006: Windows Vista Protected Video Path: unsigned software
\item 2006: Advanced Access Content System: HD DVD and Blu-ray
\item Dec 2006: ``muslix64'' publishes open source decrypting utility
\item Jan 2007: Hackers publish AACS keys
\item Apr 2007: AACS licensing adminstration revokes keys
\item Apr 2007: Motion Picture Association of America: cease \& desist
\item May 2007: Hackers publish new AACS keys 
\item May 2007: Digg removes posts; keys cover Digg; Digg gives in
\item 2008, 2009, etc.: Hackers publish new AACS keys
\end{itemize}
\pause
\begin{center}
Why has music gone DRM free but movies haven't?
\end{center}
\end{frame}

\begin{frame}{Exemption from Digital Millennium Copyright Act}
Copyright Office granted exemptions:
\begin{itemize}
\item 2000-2005: lists of websites blocked by filtering applications
\item 2000-2003: literary works/programs with failed/obsolete controls
\item 2003-\the\year: literary works/programs with failed/obsolete dongle
\item 2006-\the\year: e-books to allow screen readers
\item 2006-2009: sound recordings if there are security flaws
\item 2010-\the\year: movie clips: education, documentary, noncommercial
\item 2010-\the\year: mobile phone apps for interoperability
\item 2010-\the\year: security testing of video games
\end{itemize}
\end{frame}

\begin{frame}{Other Examples of DRM}
What types of DRM have you encountered? \\
\bigskip
Did they achieve their purpose? At a reasonable cost?\\
\bigskip
\pause
\begin{itemize}
\item Online authentication to play computer games
\item Incomplete game distributed; online connection to complete
\item Detected tampering $\rightarrow$ invincible foes (Serious Sam 3)
\item Amazon's remote deletion in 2009 of ``Nineteen Eighty-Four''
\end{itemize}
\end{frame}

\begin{frame}[fragile]{W3C: Encrypted Media Extensions (Draft, May 2013)}
%http://www.w3.org/TR/encrypted-media/
\footnotesize
\begin{lstlisting}
partial interface HTMLMediaElement {
  readonly attribute MediaKeys keys;
  void setMediaKeys(MediaKeys mediaKeys);
  attribute EventHandler onneedkey;
};
[Constructor (DOMString keySystem)]
interface MediaKeys {
  readonly attribute DOMString keySystem;
  MediaKeySession createSession(DOMString? type,Uint8Array? initData);
  static bool isTypeSupported(DOMString type,DOMString keySystem);
};
interface MediaKeySession : EventTarget {
  readonly attribute MediaKeyError? error;
  readonly attribute DOMString keySystem;
  readonly attribute DOMString sessionId;
  void update(Uint8Array key);
  void close();
};
\end{lstlisting}
%partial interface HTMLSourceElement {
%  attribute DOMString keySystem;
%};
\end{frame}

\begin{frame}{EME issues}
\href{https://www.eff.org/deeplinks/2013/10/lowering-your-standards}{Electronic Frontier Foundation}:
\begin{itemize}
\item Access control already handled by HTTPS, certificates, etc.
\item Users should have full control over their devices
\item Any played media can be recorded
\item Bad for interoperability; business negotiations to implement client
\item Alternatives: paywalls, watermarks, ads, subscriptions, etc.
\item Interferes with hyperlinking, archiving, web-spidering, etc.
\item Slippery slope from media to other HTML elements
\end{itemize}
\href{http://www.w3.org/blog/2013/05/perspectives-on-encrypted-medi/}{W3C (Jeff Jaffe)}:
\begin{itemize}
\item Institutions have right to limit access to proprietary information
\item W3C goal is to identify common needs and provide solutions
\item Support already exists for proprietary video codecs
\item If not supported, there may be no web access to resources
\end{itemize}
\end{frame}

\end{document}
