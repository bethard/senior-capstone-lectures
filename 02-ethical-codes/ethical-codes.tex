\documentclass{beamer}
\usepackage{times}
\usepackage{hyperref}
\usepackage{enumitem}

\mode<presentation>
{
  \usetheme{UAB}
  \setbeamercovered{transparent}
  \setbeamertemplate{headline}{}
  \setbeamertemplate{blocks}[rounded]
  \setbeamertemplate{navigation symbols}{}
}

\DeclareMathOperator*{\argmin}{argmin}
\DeclareMathOperator*{\argmax}{argmax}

\author{Steven Bethard}
\subject{CS 499: Senior Capstone}


\title[Ethical Codes]{Ethical Codes}
\date{August~$29^{\text{th}}$,~2013}

\begin{document}

\begin{frame}
\titlepage
\end{frame}

\begin{frame}{Quiz}

\begin{block}{1. A utilitarian ethical code is...}
\begin{enumerate}[(A)]
\item<1-3> $\displaystyle \argmax_{a \in actions} \left( \sum_{h \in humans} \left( happiness(h, a) - unhappiness(h, a)\right)\right)$
\item<1> $\displaystyle \argmax_{a \in actions} \left( \sum_{h \in humans} happiness(h, a)\right)$
\item<1> $\displaystyle \argmax_{a \in actions} \left( \sum_{h \in humans} \left( unhappiness(h, a) - happiness(h, a)\right)\right)$
\end{enumerate}
\end{block}
\begin{block}{2. Kant believed...}
\begin{enumerate}[(A)]
\item<1-2> We have a direct duty to avoid torturing animals
\item<1-2> Humans may be used as a means to an end
\item<1-3> Criminal punishment is a form of respect for the criminal
\end{enumerate}
\end{block}
\end{frame}

\begin{frame}{Utilitarianism}
Actions judged by impact on happiness of everyone
\begin{itemize}
\item Intent of action is irrelevant
\item Only consequences matter
\end{itemize}
\bigskip
Summary:
\begin{itemize}
\item $\displaystyle \argmax_{a \in actions} \left( \sum_{h \in humans} \left( happiness(h, a) - unhappiness(h, a)\right)\right)$
\end{itemize}
\end{frame}

\begin{frame}{Utilitarianism: History}
Jeremy Bentham, 1789
\begin{itemize}
\item \textit{Nature has placed mankind under the governance of two sovereign masters, pain and pleasure\ldots
By the principle of utility is meant that principle which approves or disapproves of every action whatsoever according to the tendency it appears to have to augment or diminish the happiness of the party whose interest is in question\ldots}
\item Pleasure/pain measured by: intensity, duration, certainty, propinquity, fecundity, purity, extent
\end{itemize}
\end{frame}

\begin{frame}{Utilitarianism: History}
John Stuart Mill, 1861
\begin{itemize}
\item On why happiness = good: \textit{\ldots we have not only all the proof which the case admits of, but all which it is possible to require, that happiness is a good: that each person's happiness is a good to that person, and the general happiness, therefore, a good to the aggregate of all persons.}
\end{itemize}
G. E. Moore, 1912
\begin{itemize}
\item On why happiness $\neq$ good: \textit{It involves our saying that, even if the total quantity of pleasure in each [world] was exactly equal, yet the fact that all the beings in the one possessed in addition knowledge of many different kinds and a full appreciation of all that was beautiful or worthy of love in their world, whereas none of the beings in the other possessed any of these things, would give us no reason whatever for preferring the former to the latter.}
\end{itemize}
\end{frame}

\begin{frame}{Example: Build a new Highway stretch}
Scenario:
  \begin{itemize}
  \item State may replace a curvy stretch of highway
  \item New highway segment 1 mile shorter (benefit)
  \item 150 houses would have to be removed (cost)
  \item Some wildlife habitat would be destroyed (cost)
  \end{itemize}
\pause
Analysis:
  \begin{itemize}
  \item[-] \$20 million to compensate homeowners
  \item[-] \$10 million to construct new highway
  \item[-] Lost wildlife habitat worth \$1 million
  \item[+] \$39 million savings in automobile driving costs
  \end{itemize}
Conclusion:
\begin{itemize}
\item \$39 million > \$31 million $\Rightarrow$ Building highway is a good action
\end{itemize}
\end{frame}

\begin{frame}{Evaluating Utilitarianism}
\begin{itemize}
  \item[+] Considers all stakeholders
  \item[+] Emphasizes the greatest good (i.e., happiness)
  \item[+] Cost/benefit analysis is a well-understood tool
  \item[+] Understandable ethical theory
\pause
  \item[-] Must be able to exactly quantify ``good''
  \item[-] Must be able to accurately predict all consequences
  \item[-] Majority outnumbers minority
  \item[-] Every situation has to be individually analyzed
\end{itemize}
\end{frame}

\end{document}
