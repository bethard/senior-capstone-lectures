% !TEX TS-program = pdflatexmk

\documentclass{beamer}
\usepackage{times}
\usepackage{hyperref}
\usepackage{enumitem}

\mode<presentation>
{
  \usetheme{UAB}
  \setbeamercovered{transparent}
  \setbeamertemplate{headline}{}
  \setbeamertemplate{blocks}[rounded]
  \setbeamertemplate{navigation symbols}{}
}

\DeclareMathOperator*{\argmin}{argmin}
\DeclareMathOperator*{\argmax}{argmax}

\author{Steven Bethard}
\subject{CS 499: Senior Capstone}


\title[Introduction to Computer Ethics]{CS 499 Senior Capstone: \\ Introduction to Computer Ethics}
\date{}

\begin{document}

\begin{frame}
\titlepage
\end{frame}

\begin{frame}{What is Ethics?}
\begin{itemize}
\item Laws?
\bigskip
\item Religious beliefs?
\bigskip
\item Standards of society?
\end{itemize}
\end{frame}

\begin{frame}{Ethics}
Provides a set of concepts and principles that guide us in determining what behavior helps or harms sentient creatures\\
\bigskip
\bigskip
Ethical decisions are:
\begin{itemize}
\item Rational
\item Optimal
\item Appropriate
\end{itemize}
\end{frame}

\begin{frame}{Ethical analysis can\ldots}
\begin{itemize}
\item provide a structured way to evaluate an issue and choose a course of action
\bigskip
\item help to illuminate multiple sides of the issue
\bigskip
\item help to produce persuasive arguments
\bigskip
\item help with difficult situations in professional life 
\end{itemize}
\end{frame}

\begin{frame}{Example Ethical Problem}
Is it ethical to design technology that will take jobs away from humans?
\end{frame}

\begin{frame}{History of Computer Ethics: Wiener}
Norbert Wiener, \emph{Cybernetics} (1948), \emph{The Human Use of Human Beings} (1950), \emph{God and Golem, Inc.} (1963)
\begin{itemize}
\item Topics: computers and religion, merging of human bodies with machines, robots vs. unemployment, etc.
\item Theory: humans are ``information objects'', persistent patterns of information and information processing, purpose in life is to maximize information potential
\end{itemize}
\end{frame}

\begin{frame}{History of Computer Ethics: Maner}
Walter Maner (1976) ``computer ethics''
\begin{itemize}
\item Ethical problems ``aggravated, transformed or created by computer technology''
\item Taught first computer ethics course
\begin{itemize}
\item privacy and confidentiality, computer crime, computer decisions, etc.
\end{itemize}
\item Believed computers created some wholly unique ethical problems
\end{itemize}
\end{frame}

\begin{frame}{History of Computer Ethics: Johnson}
Deborah Johnson, \emph{Computer Ethics} textbook (1985)
\begin{itemize}
\item Computers only give ``a new twist'' to existing ethics problems
\item Topics: ownership of software and intellectual property, computing and privacy, etc.
\item Topics in (1994 and 2001): hacking, technology for disabilities, impact on democracy
\end{itemize}
\end{frame}

\begin{frame}{History of Computer Ethics: Moor}
James Moor, \emph{What Is Computer Ethics?} (1985)
\begin{itemize}
\item Computers allow us to do countless things we could not do before
\item No existing laws, standards, etc. for these things
\item Computing policies should\ldots
\begin{itemize}
\item Not be among those universally considered unjust
\item Prefer policies with beneficial consequences
\end{itemize}
\end{itemize}
\end{frame}

\begin{frame}{History of Computer Ethics: Gotterbarn}
Donald Gotterbarn, \emph{Computer Ethics: Responsibility Regained} (1991)
\begin{itemize}
\item Ethics for day-to-day activities of computing professionals
\end{itemize}
\bigskip
Other Gotterbarn work:
\begin{itemize}
\item ``ACM Code of Ethics and Professional Conduct''
\item ``Software Engineering Code of Ethics and Professional Practice''
\item Licensing standards for software engineers
\end{itemize}
\end{frame}

\begin{frame}{History of Computer Ethics: G\'{o}rniak}
Krystyna G\'{o}rniak-Kocikowska, \emph{The Computer Revolution and the Problem of Global Ethics} (1995)
\begin{itemize}
\item Computer ethics will evolve into a global ethic
\item Global ethic will replace old ``local'' ethics
\end{itemize}
\end{frame}

\begin{frame}{History of Computer Ethics: Floridi}
Luciano Floridi, \emph{Information Ethics: Its Nature and Scope} (2006)
\begin{itemize}
\item Information ethics will not replace, but supplement, traditional ethics
\item ``The infosphere'': everything that exists (including humans) is an information object or process
\item Ethics must avoid damage/destruction of information
\end{itemize}
\end{frame}

\begin{frame}{History of Computer Ethics: Organizations}
\begin{itemize}
\item Computer Professionals for Social Responsibility (www.cpsr.org)
\item Electronic Frontier Foundation (www.eff.org)
\item Special Interest Group on Computing and Society (SIGCAS) of the ACM
\item Australian Institute of Computer Ethics (www.auscomputerethics.com)
\item Various conferences and journals: ETHICOMP, CEPE, EIT, IRIE, IJTHI, etc.
\end{itemize}
\end{frame}

\end{document}
