% !TEX TS-program = pdflatexmk

\documentclass{beamer}
\usepackage{times}
\usepackage{hyperref}
\usepackage{enumitem}

\mode<presentation>
{
  \usetheme{UAB}
  \setbeamercovered{transparent}
  \setbeamertemplate{headline}{}
  \setbeamertemplate{blocks}[rounded]
  \setbeamertemplate{navigation symbols}{}
}

\DeclareMathOperator*{\argmin}{argmin}
\DeclareMathOperator*{\argmax}{argmax}

\author{Steven Bethard}
\subject{CS 499: Senior Capstone}


\title{Open Source Software Development}
\date{}

\begin{document}

\begin{frame}
\titlepage
\end{frame}

\begin{frame}{Quiz}
The biggest advance to the fetchmail program came from
\begin{enumerate}[(A)]
\item<1> switching from fetchpop code to popclient code
\item<1> frequent releases and user bug reports
\item<1-2> a user patch for SMTP forwarding
\item<1> generalizing the code for multidrop support
\end{enumerate}
\medskip
Eric Raymond advises that:
\begin{enumerate}[(A)]
\item<1> Good programmers know what to rewrite and reuse. Great ones know how to write it themselves.
\item<1> Recruiting co-developers is your least-hassle route to rapid code improvement and effective debugging.
\item<1-2> Given enough beta-testers and co-developers, any problem can be characterized quickly with the fix being obvious to someone.
\item<1>  Smart code and dumb data structures works better than the other way around.
\end{enumerate}
\end{frame}

\begin{frame}{Discussion on Code Collaboration}
Give one example of how you collaborated on code.\\
\bigskip
\bigskip
What worked? What didn't?\\
\bigskip
\bigskip
\pause
For example:
\begin{itemize}
\item Division of labor between coders
\item Peer review of each other's code
\item Merging changes from multiple coders
\item etc.
\end{itemize}
\end{frame}

\begin{frame}{A Brief History of Open Source Development}
\begin{itemize}
\item TCP/IP (1982)
\item The GNU Manifesto, Richard Stallman (1985)
\item Linux kernel created, Linus Torvalds (1991)
\item ``The Cathedral and the Bazaar'' Eric S. Raymond (1997)
\item Experimental GNU Compiler System replaces GCC (1997-1999)
\item Netscape is open-sourced (1998), Mozilla is released (2002)
\end{itemize}
\end{frame}

\begin{frame}{Open Source Development Principles}
\begin{itemize}
\item Openness
\begin{itemize}
\item Anyone can see the source
\item Anyone can give feedback
\item Anyone can see feedback of others
\end{itemize}
\item Transparency
\begin{itemize}
\item Project design plans and schedules
\item Defect tracking system
\end{itemize}
\item Rapid prototyping
\begin{itemize}
\item Release early and often
\item Incorporate patches/feedback frequently
\end{itemize}
\item Community
\begin{itemize}
\item Work together for a common purpose
\item Everyone's contributions are equal
\item Best ideas (as judged by community) win
\end{itemize}
\end{itemize}
\end{frame}

\begin{frame}{Example Open Source Project}
\href{http://getbootstrap.com/}{Bootstrap} (http://getbootstrap.com/)
\begin{itemize}
\item Where is the source code?
\item How can I give feedback?
\item Where is the project plan?
\item How often do they release?
\item How large is the community?
\end{itemize}
\end{frame}

\begin{frame}{Distributed Version Control Systems}
Traditional centralized version control (CVS, SVN, etc.)
\begin{itemize}
\item Server holds a single central copy of the code
\item Clients ``commit'', recording a change in the central copy
\end{itemize}
\medskip
Distributed version control (Git, Mercurial, Bazaar, etc.)
\begin{itemize}
\item Every client has a full copy (``clone'') with all history
\item Changes can be pulled or pushed between any clones
\pause
\item[+] Most actions are fast (because they're local)
\item[+] Most actions do not require an internet connection
\item[+] Commits can be local
\item[+] Changes can be shared with selected individuals
\item[+] Branches are fast and cheap
\item[-] More time is required for the initial clone
\item[-] More space is required on the client
\end{itemize}
\end{frame}

\begin{frame}{Distributed Version Control Systems and Open Source}
\href{http://www.youtube.com/watch?v=4XpnKHJAok8&t=8m30s}{
``\ldots if you actually like using CVS\ldots you should be in some mental institution\ldots''}
\hfill -- Linus Torvalds, creator of Linux and the Git DVCS \\
\bigskip
\begin{itemize}
\item Openness
\begin{itemize}
\item Typically web-hosted and visible
\item Github, Bitbucket are free if your code is publicly visible
\end{itemize}
\item Transparency
\begin{itemize}
\item Easy tracking of evolution of features across many developers
\end{itemize}
\item Rapid prototyping
\begin{itemize}
\item Cheap branches allows many different approaches
\end{itemize}
\item Community
\begin{itemize}
\item All clones are equal (though one may still be the canonical copy)
\item Better merge support for handling many developers
\end{itemize}
\end{itemize}
\end{frame}


\end{document}
